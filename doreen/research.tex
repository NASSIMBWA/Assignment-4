\documentclass[11pt]{article}
\usepackage{cite}
\begin{document}
\begin{titlepage}
\textbf{MAKERERE UNIVERSITY\\COLLEGE OF COMPUTING AND INFORMATICS\\}
{ Nassimbwa Doreen\\ 216004538\\ 16/U/10016/EVE}

\end{titlepage}
\newpage

\textbf{A LITERATURE REVIEW ON INTERACTIVE MEDIA Ads OF MARKETING AND FACEBOOK}


\section{Introduction:}
Information and data transfer is processed faster and smartphones, laptops or tablets replace telephones, computers or books (Zhu and Chen, 2015). According to this development social media became a big importance in the last 10 years. Platforms like LinkedIn, Twitter, Instagram and Facebook are taking an essential part of our lives and they are adopted by the majority of the society (Alalwan et. al., 2016). Furthermore people have never been using social media that much than today. The overall mindset towards social media is mostly positive and people are enjoying \cite{montgomery2009interactive}. Social media users are sharing their thoughts, spreading facts, liking posts and create their own virtual user account. Worldwide there are more than 1.5 million brand pages and the number is still growing every day (Jeanjean, 2012).
\section{Body}
Social media platforms like Facebook have quickly become more popular as well more utilized by private users and business organizations. Consumer marketing and brand management on the other hand are connected with much more complexibility in this global, quickly growing technology-world. Besides that, both topics are connected more and more together. The purpose of this literature review therefore is to get a wider overview about the field of “Facebook and marketing.”By taking 61 articles \cite{de2012popularity}this study provides an overall view about different perspectives and trends. Thus it covers the role of companies, advertising and brand pages taking within Facebook and marketing.
Keywords: Face book, marketing, social media, online marketing, brand pages, users\cite{kirtics2011or}
\section{Conclusion}
The scope of the task was to review the literature on “Facebook and Marketing”. The only restriction the authors got was to mainly use scientific articles from the databases of “Web of Science”, “Scopus” and the online-library of Halmstad\cite{vanboskirk2011us} University. The majority of the articles were published in the last five years, which shows the growing importance of this subject.
\newpage
\bibliographystyle{IEEEtran}
\bibliography{references}
\end{document}
